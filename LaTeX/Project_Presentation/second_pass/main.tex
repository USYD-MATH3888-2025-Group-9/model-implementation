\documentclass[8pt]{beamer}
\usepackage[utf8]{inputenc}
\usepackage[T1]{fontenc}
\usepackage{lmodern}
\usetheme{Boadilla}
\usepackage{amsmath}
\usepackage{subcaption}
\usepackage[backend=bibtex, style=verbose]{biblatex}
\addbibresource{bib.bib}
\newcommand{\tu}[2]{#1_{\text{#2}}}


\begin{document}
	\author{Albert Stark, Calvin Ming En Kok, Kyle George, Tim Lin}
	\title{Proposal for MATH3888 project}
	\subtitle{}
	\logo{
		%\includegraphics[width=2cm]{usydlogo.png}
	}

	
	\institute{University of Sydney}
	\date{\today}
	\subject{MATH3888}
	\setbeamercovered{transparent}
	\setbeamertemplate{navigation symbols}{}

	
	\begin{frame}
	\centering
	\frametitle{}
	\maketitle
	
	\normalsize
		\begin{tabular}{ll}
		Albert Stark (SID: 540657726) & Slide 3,4: Submodels \& Comparative Analyses \\
		Calvin (SID: 530263092) & Slide 1: Introduction \\
		Kyle George (SID: 520417425) & Slide 2: The Hodgkin-Huxley and Hindmarsh-Rose models\\
		Tim Lin (SID: 530532035) & Slide 5: Conclusion \& bibliography \\
		\end{tabular}
	
	\end{frame}



	
	\begin{frame}
		\frametitle{The Hodgkin-Huxley and Hindmarsh-Rose Models}
        The Hodgkin-Huxley model is:
        \begin{align*}
            C_m\frac{dv}{dt}&=-\tu{\overline{g}}{K}n^4(v-\tu{v}{K})-\tu{\overline{g}}{Na}m^3h(v-\tu{v}{Na})-\tu{\overline{g}}{L}(v-\tu{v}{L})+\tu{I}{app}\\
            \frac{di}{dt}&=\alpha_i(1-i)-\beta_i i\quad \text{for each }i\in \{m,h,n\}
        \end{align*}
        where \(\alpha_m,\alpha_n\) are of the form \((a-v)/e^{(b-v)/c}-1\), \(\beta_m,\beta_n,\alpha_h\) are \(ae^{-bv}\) and \(\beta_h\) is of the form \(1/e^{(b-v)/c}+1\). The original values of the parameters are \(\tu{\overline{g}}{Na}=120\), \(\tu{\overline{g}}{K}=36\), \(\tu{\overline{g}}{L}=0.3\), \(C_m=1\).
        Below is the three-variable submodel of the above given by Rose and Hindmarsh:
        \begin{align*}
            \frac{dv}{dt}&=y-av^3+bv^2+I-z\\
            \frac{dy}{dt}&=c-dv^2-y\\
            \frac{dz}{dt}&=r(s(v-x_1)-z)
        \end{align*}
        where \(y=\delta w\) and \(w=r+z\) for a membrane potential \(v\), \(r\) recovery variable and \(I\) a non-constant external current. A steady state \(x_1\) is included to ensure consistency with a two-variable system ignoring \(w\). \(a,b,c,d,s,\delta\) are all constants. This model constitutes replacing the constant \(\tu{I}{app}\) with a variable \(I\) and using terms cubic in \(v\), rather than \(m\) and \(n\). This gives five points of interest in bifurcation.
	\end{frame}
    \begin{frame}{The Connor-Stevens Model}
        The Connor-Stevens ion channel model depicts a series of ion channels, with different behaviours over an index \(j\in\{1,2,3,4\}\) corresponding to \(\text{K}^+\) (\(j=2\))\(,\text{Na}^+,\) (\(j=3\)) a leaky channel (\(j=1\)) and a combined \(\text{K}^+\) and \(\text{Na}^+\) channel (\(j=4\)).
        \begin{equation*}
            I(t)=C_m\frac{dv}{dt}+\sum_{j=1}^4 \overline{g}_j (A_j(V,t))^jB_j(V,t)(V-V_j)
        \end{equation*}
        The numerical assignment \(j\) directly corresponds to the quantity of each channel in a complete set within in the cell.
        Each channel over \(j\) has its own representative \(A_j\) and \(B_j\) functions:
        \begin{align*}
            \tau_{Aj}\frac{dA_j(V,t)}{dt}+A_j(V,t)&=A_j(V,\infty)\\
            \tau_{Bj}\frac{dB_j(V,t)}{dt}+B_j(V,t)&=B_j(V,\infty)
        \end{align*}
        which are in turn dependent on constant \(\tau\) values and their steady states at infinite time \(A_j(V,\infty)\) and \(B_j(V,\infty)\). Depending on initial conditions, the model could have nine bifurcation points of interest, or it can be reduced to the Hodgkin-Huxley model with four and to the Hindmarsh-Rose model with five.
    \end{frame}
	\begin{frame}
		\frametitle{Submodels}
        There are many submodels to the Hodgkin-Huxely\footcite{hodgkinhuxely} equations, these include things like the Izhikevich\footcite{izhikevich} model for neuronal spiking, the Moris-Lecar model\footcite{morislecar}, which was designed for $Ca^{2+}$ and $K^+$ oscillations, and the Hindmarsh-Rose model \footcite{hindmarshrose}, which was also designed to model neuronal spiking.\\
        \thinspace\\
        These are all examples of models that are derived from the Hodgkin-Huxely model for the purpose of speciality. The potential of comparing these and other various derivations for this project is interesting; not least for furthering our understanding of the different situations in which to use these models, as well as to further classify the physicality of these models in alternative contexts.\\
        This analysis could be done in terms of seeing wether these models, when set up to model the same system, converge in the same time, have similar cycles, bifurcate at different points etc,etc, and to then compare which model best suits a particular situation.\\
        \thinspace\\
        Additionally, if in the course of this analysis we would be able to derive alternatively specialised models that would certainly be extremely interesting.
	\end{frame}

	\begin{frame}
		\frametitle{Comparative analyses}
        \begin{figure}
            \centering
            \begin{subfigure}{0.3\textwidth}
                \includegraphics[width=\textwidth]{2025-09-10-174547_463x477_scrot.png}
                \caption{Figure from the paper of Hindmarsh and Rose \footcite{hindmarshrose} comparing the models result (top) and experimental data (bottom)}
            \end{subfigure}
            \hfill
            \begin{subfigure}{0.3\textwidth}
                \includegraphics[width=\textwidth]{Screenshot 2025-09-10 at 18.11.37.png}
                \caption{Figure of the Hodgkin-Huxley paper \footcite{hodgkinhuxely} comparing experimental data versus that generated by their model}
            \end{subfigure}
            \hfill
            \begin{subfigure}{0.3\textwidth}
            \includegraphics[width=\textwidth]{2025-09-10-180741_636x270_scrot.png}
            \caption{Figure from the Moris-Lecar\footcite{morislecar} paper comparing experimental data (solid line) and their models result (dotted line)}
            \end{subfigure}
        \end{figure}

        All of these figures from their respective initial papers demonstrate the high suitability and accuracy of these models, comparing these models and their usability is of interest.
        
	\end{frame}


	\begin{frame}
		\frametitle{Conclusion and bibliography}
		\begin{tiny}
			\printbibliography
		\end{tiny}
	\end{frame}
\end{document}
