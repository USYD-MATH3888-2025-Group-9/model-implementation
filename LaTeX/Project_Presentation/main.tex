\documentclass[8pt]{beamer}
\usepackage[utf8]{inputenc}
\usepackage[T1]{fontenc}
\usepackage{lmodern}
\usetheme{Boadilla}
\usepackage{circuitikz}

\usepackage[backend=bibtex, style=alphabetic]{biblatex}
\addbibresource{bib.bib}



\begin{document}
	\author{Albert Stark, Calvin Ming En Kok, Kyle George, Tim Lin}
	\title{Proposal for MATH3888 project}
	\subtitle{}
	\logo{
		%\includegraphics[width=2cm]{usydlogo.png}
	}

	
	\institute{University of Sydney}
	\date{\today}
	\subject{MATH3888}
	\setbeamercovered{transparent}
	\setbeamertemplate{navigation symbols}{}

	
	\begin{frame}
  \centering
  \frametitle{Inhibitor-Induced Wavetrains and Spiral Waves \\ 
  in an Extended FitzHugh--Nagumo Model} \\[1em]
\begin{figure}
    \centering
    \includegraphics[width=0.5\linewidth]{Screenshot 2025-09-03 145506.png}
    \caption{Travelling and spiral waves in an extended FitzHugh–Nagumo system.}
    \label{fig:placeholder}
\end{figure}
  \normalsize
  \begin{tabular}{ll}
    Albert Stark (SID: 540657726) & Slide 4: Experimental Data \\
    Calvin (SID: 530263092) & Slide 1,5: Introduction and Conclusion \\
    Kyle George (SID: 520417425) & Slide 2: Model Equations \\
    Tim Lin (SID: 530532035) & Slide 2: Biology / Physiology \\
  \end{tabular}


\end{frame}
	
	\begin{frame}
		\frametitle{The FitzHugh-Nagumo model}
        \begin{align*}
		\varepsilon \frac{dv}{dt}&=f(v)-w+I_{\text{app}}&\text{(cubic)}
		\\
		\frac{dw}{dt}&=v-\gamma w &\text{(linear)}
		\end{align*}
		where
		\[
		f(v)=v(1-v)(v-\alpha)
		\]
		for typical values within \(0<\alpha<1,\varepsilon\ll 1,\gamma\approx0.5\) \cite{GaniM.Osman2022IWaS}\cite{LabouriauIsabelSalgado2014Psia}. This simplifies the four Hodgkin-Huxley equations to two: one fast and one slow. The circuit diagram below has equivalent function \cite{Textbook}.
		\begin{center}
			\begin{circuitikz}
			\draw (0,0) to [short,o-,l^=$V_e$] (1,0);
			\draw (1,0) to (1,1) to [C,l_=C] (3,1) to (9,1) to (9,0);
			\draw (1,0) to (3,0) to [R,l_=R] (5,0) to [L,l_=L] (7,0) to [battery,l_=$V_0$] (9,0) to [short,-o,l^=$V_i$] (10,0);
			\draw (1,0) to (1,-1) to [vR,l_=f(v)] (3,-1) to (9,-1) to (9,0);
			\end{circuitikz}
		\end{center}
	\end{frame}

	\begin{frame}
		\frametitle{Biology}
        \begin{itemize}
            \item Excitability: is a ability of a cell, particularly nerve and muscle cells, to respond to a stimulus by generating an 
            electrical signal called an action potential.
            \item It is an intrinsic membrane property that relies on rapid ion fluxes across the cell membrane to change the membrane 
            potential to a "threshold potential," triggering the signal; where threshold potential is the minimum potential change required to 
            trigger an action potential.
		    \item The model explains how small inputs decay, but once threshold is crossed, a large spike occurs (like a real neuron). After a 
            spike, the recovery variable w prevents immediate re-excitation (models absolute/relative refractory period).
  
        \end{itemize}
       
	\end{frame}

	\begin{frame}
		\frametitle{Experimental Data}
		%talk about experimental data from real live stuff <-> compare with comp model
		%modification of FHN model in the paper, particularly with respect to the new addition of a function G, on which the paper is p
		%The FitzHugh-Nagumo equations are modified in the referred paper as follows:
		%\[
		%\begin{cases}
		%				\frac{\partial u}{\partial t} = d_u (\frac{\partial^2 u}{\partial x^2} + \frac{\partial^2 u}{\partial x^2}) + u(1-u)(u-1)-v &\rightarrow \text{ the same}\\
		%				\thinspace\\
		%				\frac{\partial v}{\partial t} = d_v (\frac{\partial^2 u}{\partial x^2} + \frac{\partial^2 u}{\partial x^2}) + \epsilon(u-\gamma u) &\rightarrow d_v (\frac{\partial^2 u}{\partial x^2} + \frac{\partial^2 u}{\partial x^2}) + 
		%				G_b(u-\gamma u)
		%\end{cases}
		%\]
		%There are certain conditions on this new $G_b(x)$, in particular it is obtained from the equation, with some additional conditions
		In general we are restricted to a certain class of functions $G_b(x)$ that satisfy the following conditions:
		\[
		G_b(x) = \tfrac{G(bx)}{b} , b \in \mathbb{R^+}\quad G(x) \text{ satisfies }
		\begin{cases}
			G(x) = 0 \iff x = 0\\
			G(x) > 0 \iff x > 0, \quad G(x) < 0 \iff x < 0\\
			\lim_{x\rightarrow-\infty} (G(x) - x) = 0, \quad	\lim_{x\rightarrow\infty} G(x) \geq 0
		\end{cases}
		\]
		\[\text{examples: }G_b(x) := \tfrac{x e^{-x}}{e^x + e^{-x}} \text{ (non-monotone)} \quad G_b(x) := \tfrac{x-\sqrt{x^2 - 4}}{2} + 1 \text{ (monotone)}\]
%		\vspace{0.1cm}

		The paper demonstrates experimentally (via computational simulation) the existence of spiral waves in both the examples provided. \quad The wavetrains were identified with the use of AUTO in the paper, this is possible for us, although python alternatives such as PyCoBi \cite{PyCoBiPackage} and scFates \cite{scFatesPaper} \cite{scFatesGithub} are also of interest.\\
		
		The potential of characterising in general the monotone or non-monotone functions that generate these spiral waves is of interest, as is the physicality of these functions and their solutions.\\
		\thinspace\\
		
		We anticipate that we will perform wavetrain and spiral-wave analysis. As well as bifurcation and limit-cycle analysis

	\end{frame}


	\begin{frame}
		\frametitle{Conclusion + bib}
		\begin{tiny}
			\printbibliography
		\end{tiny}
	\end{frame}

%	\frame[shrink=30]{\printbibliography}
\end{document}
